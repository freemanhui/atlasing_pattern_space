% arxiv-preprint.tex — Pandoc template (pdfLaTeX-compatible, arXiv-safe)
\documentclass[11pt]{article}

% --- Encoding & fonts (pdfLaTeX) ---
\usepackage[utf8]{inputenc}
\usepackage[T1]{fontenc}
\usepackage{lmodern}
\usepackage{microtype}

% --- Page layout ---
\usepackage{geometry}
\geometry{margin=1in}

% --- Math & figures ---
\usepackage{amsmath, amssymb}
\usepackage{graphicx}
\usepackage{booktabs}

% --- Hyperlinks ---
\usepackage{hyperref}
\hypersetup{
  colorlinks=true,
  linkcolor=blue,
  citecolor=blue,
  urlcolor=blue
}

% --- Author/affiliation block (optional) ---
\usepackage{authblk}

% --- Bibliography options ---
%  \usepackage[sort&compress,numbers]{natbib}

% --- Title metadata ---
  \title{When Does Causal Regularization Help? A Systematic Study of Boundary Conditions in Spurious Correlation Learning}
  \author{Freeman Hui}
  \date{}

% --- Define tightlist for Pandoc compatibility ---
\providecommand{\tightlist}{%
  \setlength{\itemsep}{0pt}\setlength{\parskip}{0pt}}

\begin{document}
\maketitle


\begin{abstract}
We challenge the conventional wisdom that explicit causal regularization is necessary for out-of-distribution generalization. Through systematic investigation on ColoredMNIST, we discover that reconstructive architectures like autoencoders provide a powerful \textbf{implicit causal bias} that largely obviates the need for explicit methods like IRM or HSIC. Autoencoder baselines achieve 82-86\% accuracy with 99\% spurious correlation, with explicit causal losses adding only marginal (0-4pp) gains.

Using the Atlasing Pattern Space (APS) framework---a modular toolkit combining topology preservation (T), causal invariance (C), and energy shaping (E)---we establish clear \textbf{boundary conditions} for when explicit regularization helps. Our experiments across multiple domains reveal that: (1) explicit causal methods become critical only when architectural bias is absent or spurious correlations are pathologically strong; (2) topology preservation improves kNN fidelity in high-dimensional vision tasks but fails completely in low-dimensional synthetic settings; and (3) energy-based regularization effectively prevents overfitting while maintaining OOD accuracy.

Through controlled experiments including a systematic study of component domain-specificity, we demonstrate that regularization components are not universally beneficial but rather require careful domain-specific validation. Our results reframe causal learning as a hierarchical process: architectural choice is primary, with explicit regularizers serving as targeted, domain-specific corrections when architectural bias proves insufficient.
\end{abstract}

\section{Introduction}\label{introduction}

The pursuit of models that generalize out-of-distribution (OOD) has centered on developing explicit causal regularization techniques like Invariant Risk Minimization (IRM) \cite{arjovsky2019invariant} and Hilbert-Schmidt Independence Criterion (HSIC) losses \cite{gretton2005hsic}. The underlying assumption is that standard models naively exploit spurious correlations, and that these explicit regularizers are necessary to force models to learn invariant, causal features.

This paper challenges that assumption. We demonstrate that for a broad class of models, \textbf{architectural choice can be a more powerful driver of causal learning than explicit regularization}. Specifically, we find that reconstructive models like autoencoders possess a strong \textbf{implicit causal bias}. By being forced to reconstruct the input, these models naturally learn to prioritize structural (causal) features over superficial (spurious) ones, even when the spurious features are overwhelmingly correlated with the label.

Our central finding, derived from a systematic study on ColoredMNIST, is that a standard autoencoder baseline achieves 82-86\% accuracy with 99\% spurious correlation. Explicit causal regularizers, when added, provide only marginal gains (0-4pp), suggesting they are largely redundant when a strong architectural bias is already present. This discovery reframes the central question from \emph{`How do we add causal constraints?'} to \emph{`When are they actually needed?'}

To dissect this interplay between implicit and explicit bias, we employ the \textbf{Atlasing Pattern Space (APS)} framework as a modular diagnostic toolkit. APS combines three regularizers:
\begin{itemize}
    \item \textbf{Topology (T)}: Preserves the manifold structure of the data.
    \item \textbf{Causality (C)}: Enforces invariance to nuisance factors (e.g., via HSIC).
    \item \textbf{Energy (E)}: Shapes the latent space using a data-driven energy function.
\end{itemize}

Using APS, we establish clear \textbf{boundary conditions} for causal learning. We show that explicit methods (like the C component) become critical only when the implicit architectural bias is absent (e.g., in simple feed-forward classifiers) or when the data presents pathological spurious correlations (approaching 100\%).

Beyond the architectural bias finding, our systematic experiments across vision and NLP domains reveal domain-specific boundary conditions for each APS component. We find that topology preservation improves kNN fidelity in high-dimensional vision tasks (MNIST) but fails to provide measurable benefits in low-dimensional synthetic settings, demonstrating that geometric regularization is not universally applicable. Energy-based regularization consistently prevents overfitting but provides only marginal OOD accuracy improvements, suggesting its role is primarily as a capacity control mechanism rather than a causal learning tool.

Our contributions are thus threefold:
\begin{enumerate}
    \item \textbf{Implicit causal bias discovery}: We demonstrate that architectural choice (reconstruction) is a primary mechanism for OOD generalization, challenging the default reliance on explicit regularizers.
    \item \textbf{Boundary condition characterization}: Through systematic domain-specificity experiments, we identify when and why each regularization component helps (or fails), providing practical guidelines for practitioners.
    \item \textbf{Honest negative results}: We report complete failure of topology preservation in low-dimensional synthetic domains, demonstrating that regularization components are not universally beneficial.
\end{enumerate}

Ultimately, this work advocates for a more nuanced, hierarchical approach to causal learning: begin with the right architecture, and only then apply explicit regularization as a targeted, second-order correction.

\textbf{Motivation:} The name \emph{``Atlasing''} evokes the creation of
a map or atlas of all patterns (e.g. linguistic or visual patterns) such
that distance and neighborhoods on the map reflect true semantic or
functional similarity. Unlike standard embedding methods which largely
treat latent dimensions as unstructured, APS treats representation
learning as a \textbf{manifold learning problem} with additional causal
and energy-based regularization. By doing so, APS aims to produce latent
``charts'' that are easier to interpret and navigate -- much like an
atlas that faithfully represents the terrain: - In NLP, an APS-learned
embedding might place synonyms or contextually similar phrases in
adjacent regions (topology), align dimensions with abstract concepts
(causality), and form energy basins for distinct topics or themes
(energy). - In computer vision, APS could map images such that images
with similar content or style cluster together (topology), latent
variables isolate factors like lighting or viewpoint (causality), and
each object category corresponds to an energy basin that stores its
prototypical patterns. - In recommendation systems, user/item embeddings
could be structured so that similar users/items lie in contiguous latent
neighborhoods, confounding factors (e.g. popularity) are factored out,
and communities or genres appear as attraction basins.

By integrating these properties, APS promises representations that
support \textbf{better generalization} (through invariant features),
\textbf{robustness to spurious correlations} (through causal structure),
and \textbf{enhanced interpretability} (through topologically and
energetically organized latent maps). In the following sections, we
formalize the APS framework and discuss related work that inspires each
component (Topology, Causality, Energy). We then outline the methodology
for implementing APS and propose experiments to evaluate its benefits.

\section{Related Work}\label{related-work}

\subsection{Topology-Preserving
Embeddings}\label{topology-preserving-embeddings}

Our emphasis on latent \textbf{topology preservation} builds on a rich
history of manifold learning and neighbor-preserving embeddings.
Classical techniques like
\textbf{t-SNE}\href{https://www.jmlr.org/papers/v9/vandermaaten08a.html\#:~:text=We\%20present\%20a\%20new\%20technique,very\%20large\%20data\%20sets\%2C\%20we}{{[}1{]}}
and
\textbf{UMAP}\href{https://arxiv.org/abs/1802.03426\#:~:text=,reduction\%20technique\%20for\%20machine\%20learning}{{[}2{]}}
aim to embed high-dimensional data into low dimensions (e.g. 2D) for
visualization, such that similar points stay close and multi-scale
structure is maintained. In particular, UMAP uses a framework from
algebraic topology to learn a low-dimensional mapping that preserves
both local and some global structure of the data
manifold\href{https://arxiv.org/abs/1802.03426\#:~:text=,reduction\%20technique\%20for\%20machine\%20learning}{{[}2{]}},
while t-SNE focuses on retaining local neighbor affinities and revealing
cluster structure at multiple
scales\href{https://www.jmlr.org/papers/v9/vandermaaten08a.html\#:~:text=We\%20present\%20a\%20new\%20technique,very\%20large\%20data\%20sets\%2C\%20we}{{[}1{]}}.
These methods underscore the value of respecting the intrinsic topology
of data, although they are typically used as post-hoc visualizers rather
than as trainable model components.

In neural network research, recent work has explicitly added topological
or geometric constraints to latent spaces. \textbf{Topological
Autoencoders} (Moor et al. 2020) introduced a differentiable loss based
on persistent homology to ensure that the topology (e.g. connectivity,
loops) of the latent space matches that of the input space. By
penalizing differences in Betti numbers and other topological features
between input and latent distributions, they preserved multi-scale
connectivity and improved interpretability of latent dimensions. Other
approaches enforce local geometric fidelity: for example, \textbf{Local
Distance Preserving Autoencoders} (Chen et al. 2022) add a loss that
keeps the distances between each point and its $k$-nearest neighbors
in data space similar in latent space. This is achieved via a continuous
$k$-NN graph that captures topological features at all scales, used as
a constraint during training. Such methods align with earlier ideas like
\textbf{Laplacian eigenmaps} and \textbf{locally linear embedding
(LLE)}, which also preserve neighbor relations in a lower-dimensional
embedding of the data manifold.

Graph-based regularization of latent geometry has shown promise in
autoencoders. For instance, \textbf{Neighborhood Reconstructing
Autoencoders (NRAE)} (Lee et al. 2021) incorporate a term ensuring that
each data point's local neighborhood (from a precomputed graph) is
reconstructed by the decoder, thus correcting ``wrong local connectivity
and geometry'' often observed in vanilla AEs. Similarly, the
\textbf{Witness Autoencoder (W-AE)} and \textbf{Geometry-Regularized
Autoencoder (GRAE)} introduced topological and geometric regularizers
(e.g. using witness complexes or manifold charts) to shape the latent
space. These works demonstrate that \textbf{imposing topology-awareness
during representation learning leads to latent spaces that better
reflect the true structure of data}, which can improve downstream tasks
and the realism of interpolations. APS adopts this principle: our
\textbf{Topology (T)} component will preserve neighborhood relationships
(e.g. via a $k$-NN graph or topological loss) so that the learned
atlas maintains the continuity and connectivity of the original pattern
space.

\subsection{Causal and Invariant Representation
Learning}\label{causal-and-invariant-representation-learning}

The \textbf{Causality (C)} component of APS seeks to make latent
features invariant to nuisance factors and aligned with stable,
meaningful properties. This idea is inspired by research in
\textbf{causal representation learning} and \textbf{domain
generalization}. A key insight from causality is that models should
capture the \emph{invariant mechanisms} underlying data rather than
spurious correlations. \textbf{Invariant Risk Minimization (IRM)}
(Arjovsky et al. 2019) formalized this by learning a data representation
such that \emph{the optimal classifier on that representation is the
same across multiple environments}. By leveraging data from different
environments (or domains), IRM encourages the encoder to discard
features that are inconsistent (spurious) and keep those that have a
stable relationship with the target, thereby improving
out-of-distribution (OOD) generalization. APS can incorporate this
principle by using multiple data contexts or augmentations and adding a
penalty if a classifier's predictions differ between contexts when using
the APS embedding.

Another line of work uses \textbf{independence criteria} to enforce
invariances. The \emph{Hilbert-Schmidt Independence Criterion} (HSIC) is
a kernel-based measure of statistical independence. It has been used as
a loss to encourage representations $Z$ to be independent of certain
variables $V$ (for example, sensitive attributes or domain labels).
Greenfeld and Shalit (2020) applied HSIC as a regularizer to achieve
robust models under covariate shift. By penalizing any dependence
between the model's residuals and the input distribution, their
HSIC-based loss yielded predictors where $Y -
\hat\{f\}(X)$ is nearly independent of $X$,
corresponding to a scenario where only the causal relation (and
independent noise) remains. In APS, we can use HSIC-based penalties to
encourage that the learned latent $Z$ is independent of nuisance
factors (e.g. style, noise, context that we want to factor out).
Similarly, other works like \emph{Domain-Adversarial Training} and
\emph{Maximum Mean Discrepancy (MMD)} have sought to remove
domain-specific information from embeddings, but HSIC offers a direct,
differentiable independence measure.

There is also overlap between invariant representation learning and
\textbf{disentangled representation learning}. Methods such as
\textbf{$\beta$-VAE} (Higgins et al. 2017) aim to learn
latent factors that correspond to independent generative factors of
variation\href{https://openreview.net/forum?id=Sy2fzU9gl\#:~:text=and\%20reason\%20in\%20the\%20same,disentangled\%20factor\%20learning\%20on\%20a}{{[}3{]}}.
By constraining the VAE's latent channel capacity (via a higher
$\beta$ weight on the KL-divergence term),
$\beta$-VAE encourages the latent dimensions to capture
distinct aspects of the data (for example, in an image dataset, one
dimension may capture ``rotation'' while another captures
``scale'')\href{https://openreview.net/forum?id=Sy2fzU9gl\#:~:text=representations\%20from\%20raw\%20image\%20data,degree\%20of\%20disentanglement\%20learnt\%20by}{{[}4{]}}.
The result is an interpretable factorized representation that is aligned
with \emph{causal factors} in the data generation process, achieved
without supervision. APS's causality module shares this goal of
\textbf{isolating meaningful factors}: through losses like IRM or HSIC
(and potentially by borrowing ideas from $\beta$-VAE to
enforce factorization), APS encourages each latent dimension or subspace
to correspond to a stable property of the input, invariant to minor
changes or context. Indeed, the broader vision of \textbf{causal
representation learning} is to uncover latent features that correspond
to real-world causal variables, a direction articulated in surveys like
Schölkopf et al. (2021) \emph{``Towards Causal Representation
Learning.''} APS contributes to this direction by integrating causal
invariance constraints directly into the representation learning
objective.

\subsection{Energy-Based Models and Attractor
Networks}\label{energy-based-models-and-attractor-networks}

The \textbf{Energy (E)} component of APS introduces an
\textbf{energy-based perspective} to the latent space. Energy-Based
Models (EBMs) assign an unnormalized ``energy'' score to configurations
(in our case, latent vectors), such that low-energy regions correspond
to probable or familiar patterns. By shaping the latent space's energy
landscape into \textbf{basins of attraction}, APS aims to create
distinct wells (valleys) that capture clusters or prototypes of
patterns. This idea is reminiscent of \textbf{Hopfield networks} and
other attractor models. A classical Hopfield network (Hopfield 1982)
stores patterns as stable fixed points of a dynamical system; when the
network state is perturbed to a new input, it iteratively updates and
converges to the nearest stored pattern (an attractor). Recent work has
modernized this concept: \emph{``Hopfield Networks is All You Need''}
(Ramsauer et al. 2021) showed that a continuous-state Hopfield layer can
store exponentially many patterns and that its update rule is equivalent
to the Transformer's attention
mechanism\href{https://arxiv.org/abs/2008.02217\#:~:text=,These}{{[}5{]}}\href{https://arxiv.org/abs/2008.02217\#:~:text=rule\%20is\%20equivalent\%20to\%20the,recurrent\%20networks\%2C\%20and\%20provide\%20pooling}{{[}6{]}}.
Importantly, they identified different types of energy minima in such
networks: global minima that average over all patterns, metastable
states averaging subsets of patterns, and fixed-point attractors
corresponding to individual stored
patterns\href{https://arxiv.org/abs/2008.02217\#:~:text=,These}{{[}5{]}}.
This suggests that deep networks can incorporate Hopfield-like memory to
perform pooling, association, and rapid content-based
retrieval\href{https://arxiv.org/abs/2008.02217\#:~:text=heads\%20perform\%20in\%20the\%20first,of\%20four\%20considered\%20multiple\%20instance}{{[}7{]}}.
APS leverages this concept by aiming for a latent space where each
significant pattern or concept acts as an \textbf{attractor}. For
example, in an NLP context, an abstract concept (like \emph{sports})
might form an energy basin that attracts semantically related sentence
embeddings, enabling the model to recall or generate prototypical
examples of that concept.

Energy-based modeling has also been applied \textbf{in the latent spaces
of generative models}. Rather than using a fixed prior (e.g. Gaussian)
in a VAE or generator, researchers have learned \textbf{latent space
EBMs} to better model complex distributions. For instance, Pang et al.
(2020) train a VAE-like generative model where the latent prior $p(z)$
is not a simple Gaussian but given by an energy-based model learned
jointly with the
decoder\href{https://proceedings.neurips.cc/paper/2020/file/fa3060edb66e6ff4507886f9912e1ab9-Paper.pdf\#:~:text=We\%20propose\%20to\%20learn\%20energy,is\%20efficient\%20and\%20mixes\%20well}{{[}8{]}}.
Their latent EBM prior, parameterized by a small network, captures the
structure of the latent codes that correspond to real data, leading to
improvements in image and text
generation\href{https://proceedings.neurips.cc/paper/2020/file/fa3060edb66e6ff4507886f9912e1ab9-Paper.pdf\#:~:text=that\%20the\%20observed\%20example\%20is,as\%20uniform\%20or\%20isotropic\%20Gaussian}{{[}9{]}}\href{https://proceedings.neurips.cc/paper/2020/file/fa3060edb66e6ff4507886f9912e1ab9-Paper.pdf\#:~:text=match\%20at\%20L36\%20latent\%20vector,can\%20be\%20learned\%20jointly\%20by}{{[}10{]}}.
Because the latent space is low-dimensional, sampling from the EBM (via
MCMC) is efficient and yields diverse samples that respect the learned
data
manifold\href{https://proceedings.neurips.cc/paper/2020/file/fa3060edb66e6ff4507886f9912e1ab9-Paper.pdf\#:~:text=that\%20the\%20observed\%20example\%20is,as\%20uniform\%20or\%20isotropic\%20Gaussian}{{[}9{]}}\href{https://proceedings.neurips.cc/paper/2020/file/fa3060edb66e6ff4507886f9912e1ab9-Paper.pdf\#:~:text=MCMC\%20sampling\%20from\%20both\%20the,is\%20efficient\%20and\%20mixes\%20well}{{[}11{]}}.
This approach essentially carves out an \textbf{energy landscape in
latent space shaped by the data}, rather than assuming latent variables
are independent. APS's energy component aligns with this strategy: by
training an energy function $E(z)$ alongside the encoder, we ensure
that latent representations of training data lie in low-energy valleys,
while high-energy barriers separate distinct pattern regions.

\textbf{However, our experimental validation revealed a critical
insight:} memory-based energy functions that create arbitrary attractor
basins \emph{compete} with topology preservation, causing catastrophic
failure (detailed in Section~\ref{sec:experiments}). This led to the
development of \textbf{TopologyEnergy}, a data-driven approach where
energy is minimized when k-NN adjacency relationships are preserved:
$$
E_{\text{topo}}(z) = -\frac{\sum_{i,j} A^{\text{orig}}_{ij} \cdot A^{\text{latent}}_{ij}}{n \cdot k}
$$
where $A^{\text{orig}}$ and $A^{\text{latent}}$ are k-NN adjacency
matrices in original and latent space. This formulation naturally
\textit{aligns} with the topology objective ($\mathcal{L}_T$) rather
than creating arbitrary basins, achieving 902\% better label alignment
(ARI) than memory-based approaches on MNIST while maintaining
reconstruction quality.

\subsubsection{Visualization of Energy Landscapes}\label{energy-visualization}

To illustrate the energy basin concept concretely, Figure~\ref{fig:energy3d} shows a 3D energy surface with four prototype basins. The low-energy valleys (shown in blue) cluster latent codes into semantic regions, with each prototype marked by a red X. This visualization demonstrates how the energy function $E(z)$ creates natural attractors in the latent space.

\textit{Note: While memory-based energy creates discrete basins as shown in Figures~\ref{fig:energy3d}--\ref{fig:energytraj}, our final implementation uses TopologyEnergy for superior performance, avoiding the catastrophic failures of arbitrary attractors (see Section~\ref{sec:experiments}).}

\begin{figure}[htbp]
  \centering
  \includegraphics[width=0.45\textwidth]{illustrated/figure1.png}
  \caption{\small 3D energy surface with four prototype basins (marked by red X's). Low-energy valleys cluster latent codes into semantic regions.}
  \label{fig:energy3d}
\end{figure}

The sharpness of these energy basins can be controlled by a temperature parameter $\beta$.

\begin{figure}[h]
  \centering
  \includegraphics[width=0.45\textwidth]{illustrated/figure2.png}
  \caption{\small Energy vs. distance for different $\beta$: sharper $\beta=10$ basins approximate Hopfield-like memory; lower $\beta=1$ yields smoother RBF-style landscapes.}
  \label{fig:energycross}
\end{figure}

Figure~\ref{fig:energytraj} demonstrates the attractor dynamics by showing trajectories of points descending the energy landscape. Each trajectory flows from an initial position toward the nearest prototype basin, illustrating how the energy function guides latent representations toward stable semantic clusters. This attractor behavior provides robustness to noise and enables memory recall: perturbed representations naturally flow back to their corresponding prototypes.

\begin{figure}[h]
  \centering
  \includegraphics[width=0.45\textwidth]{illustrated/figure3.png}
  \caption{\small Trajectories descending the energy landscape into attractor basins. Each point flows via gradient descent on $E(z)$ to the nearest prototype.}
  \label{fig:energytraj}
\end{figure}

The idea of \textbf{energy valleys aiding interpretation} can be seen
through techniques like analyzing latent vector fields. Recent studies
observe that standard training often already induces some attractor
dynamics in latent
spaces\href{https://arxiv.org/html/2505.22785v3\#:~:text=The\%20set\%20of\%20initial\%20conditions,denoted\%20as\%20basin\%20of\%20attraction}{{[}12{]}}\href{https://arxiv.org/html/2505.22785v3\#:~:text=Informally\%2C\%20this\%20implies\%20that\%20the,probability\%20on\%20the\%20data\%20manifold}{{[}13{]}}
-- autoencoders with contractive mappings can cause points to flow
towards regions of high data density (an implicit energy
model)\href{https://arxiv.org/html/2505.22785v3\#:~:text=Why\%20are\%20neural\%20mappings\%20contractive,main\%20factors\%20promoting\%20contractive\%20behavior}{{[}14{]}}\href{https://arxiv.org/html/2505.22785v3\#:~:text=We\%20build\%20on\%20this\%20by,corresponding\%20prior\%20in\%20latent\%20space}{{[}15{]}}.
APS makes this explicit and controllable. By designing $E(z)$ (or
using a Hopfield layer) we define where the attractors should be, which
can correspond to semantic categories or recurring prototypes in data.
This has practical benefits: for \textbf{generation}, one can sample
from these basins to produce novel but coherent outputs; for
\textbf{classification}, the basin a new point falls into can directly
indicate its class or type; for \textbf{anomaly detection}, points
landing in no known basin (high energy areas) are flagged as outliers.
Overall, the Energy component of APS connects to a broad trend of
integrating \textbf{EBMs and dynamical systems} with deep
learning\href{https://arxiv.org/abs/2008.02217\#:~:text=heads\%20perform\%20in\%20the\%20first,of\%20four\%20considered\%20multiple\%20instance}{{[}7{]}},
providing a bridge between pattern recognition and pattern generation
via the geometry of the latent space.

\subsection{Structured and Interpretable
Embeddings}\label{structured-and-interpretable-embeddings}

Beyond the specific T, C, and E aspects, APS relates to the general
pursuit of \textbf{structured and interpretable embeddings} in machine
learning. Traditional word embeddings (e.g. Word2Vec, GloVe) exhibit
surprising linear structure enabling analogies, but are largely learned
from distributional statistics. Follow-up analyses have shown that these
embedding spaces have meaningful directions (e.g. gender or tense
directions) but also problematic biases. By contrast, approaches that
\emph{impose} structure can yield more interpretable representations.
One notable example is \textbf{Hyperbolic Embeddings} for representing
hierarchical data. Nickel \& Kiela (2017) introduced \textbf{Poincaré
Embeddings}, which learn embeddings in a hyperbolic space (an
$n$-dimensional Poincaré ball) to naturally represent tree-like
hierarchies\href{https://papers.nips.cc/paper_files/paper/2017/hash/59dfa2df42d9e3d41f5b02bfc32229dd-Abstract.html\#:~:text=characteristic\%20for\%20many\%20complex\%20symbolic,representation\%20capacity\%20and\%20in\%20terms}{{[}16{]}}.
Thanks to the negative curvature, hyperbolic space can encode
hierarchical relationships with much lower distortion than Euclidean
space -- allowing one to capture both \textbf{similarity and hierarchy}
simultaneously\href{https://papers.nips.cc/paper_files/paper/2017/hash/59dfa2df42d9e3d41f5b02bfc32229dd-Abstract.html\#:~:text=characteristic\%20for\%20many\%20complex\%20symbolic,representation\%20capacity\%20and\%20in\%20terms}{{[}16{]}}.
They demonstrated significantly improved representation capacity and
generalization for data with latent hierarchies (like WordNet noun
relationships) when using hyperbolic embeddings as opposed to
Euclidean\href{https://papers.nips.cc/paper_files/paper/2017/hash/59dfa2df42d9e3d41f5b02bfc32229dd-Abstract.html\#:~:text=embedding\%20them\%20into\%20hyperbolic\%20space,in\%20terms\%20of\%20generalization\%20ability}{{[}17{]}}.
This is a powerful reminder that the \textbf{choice of geometry} for the
latent space can profoundly shape what structures can be efficiently
represented. APS is agnostic to a specific geometry (one could even
conceive APS on a hyperbolic manifold if the data is hierarchical), but
it shares the spirit of \emph{baking domain-relevant structure into the
embedding space}. In the case of APS, the inductive biases are
topological (neighbor relations), causal, and energy-based structure.

\textbf{Interpretable latent dimensions} are also pursued in
disentanglement research (as mentioned with $\beta$-VAE)
and in various supervised settings (e.g. learning a latent space aligned
with known attributes or concepts). In NLP, there have been efforts to
find or impose latent dimensions that correspond to semantic attributes
-- for example, latent edit vectors for style, sentiment, etc., which
can be manipulated. APS could help here by explicitly designating parts
of the latent space to capture certain factors (through the causal
invariance objective) and ensuring those parts are used consistently
across data. Furthermore, visualization techniques like \textbf{UMAP and
t-SNE} can be directly applied to APS embeddings to produce ``maps'' of
the learned pattern space, potentially revealing clear organization
(clusters, hierarchies, continuous variations) that align with
human-understandable categories. By contrast, in a standard embedding
space, such visualizations might be muddled by entangled factors or lack
of global structure. There are also alternatives like
\textbf{Topological Data Analysis (TDA)} tools (e.g. Mapper algorithm)
that could be used to assess how well APS preserves the shape of data.
Indeed, TopoGraph-based evaluation was used by Moor et al. to show
improved latent topology. We anticipate that APS embeddings will lend
themselves to clearer topological summaries and interactive exploration,
essentially acting as an atlas for researchers to \textbf{navigate the
pattern space}.

\section{Atlasing Pattern Space (APS)
Framework}\label{atlasing-pattern-space-aps-framework}

\subsection{Overview}\label{overview}

APS learns an encoder $f: X \to Z$ (and potentially a
decoder $g: Z \to X$ in an autoencoder setup) such that
the latent space $Z$ becomes an \textbf{atlas} of the data manifold
with the properties of \textbf{Topology preservation (T)},
\textbf{Causal invariance (C)}, and \textbf{Energy structuring (E)}.
These three aspects are enforced via dedicated loss terms added to the
training objective alongside any task-specific loss (e.g. reconstruction
error or prediction loss). Figure 1 (conceptual; see Appendix)
illustrates the APS concept: in latent space, points form neighborhoods
corresponding to similar inputs (T), lie on coordinate axes
corresponding to meaningful factors (C), and cluster into basins around
prototypical exemplars (E).

Formally, let $z = f(x)$ be the embedding of input $x$. APS's
training objective can be written as:
$$
\mathcal{L}_{\text{APS}} = \mathcal{L}(x, z) + \lambda_T \mathcal{L}_T(x, z) + \lambda_C \mathcal{L}_C(x, z) + \lambda_E \mathcal{L}_E(z),
$$
where $\mathcal{L}$ could be a reconstruction loss (if APS
is an autoencoder) or a classification loss (if APS is used in a
supervised setting), and $\lambda_T, \lambda_C, \lambda_E$ are weights for the
regularizers. We describe each component loss below:

\textbf{(T) Topology-Preserving Loss:}
$\mathcal{L}_{T}$ ensures that local neighborhoods
in input space $X$ are reflected in $Z$. One implementation is a
\textbf{continuous $k$-NN graph loss}: we construct a graph $G$ on
the batch (or dataset) in input space where edges connect each point to
its $k$ nearest neighbors (using original input features or a
predefined distance). We then encourage the distances in latent space
$d_Z(f(x_i), f(x_j))$ to be small for edges $(i,j)$ in $G$ and,
optionally, to be larger for non-neighbor pairs. For example, a
\textbf{triplet loss} or contrastive loss can be used:
$\mathcal{L}_T = \sum \big[\Delta - |z_i - z_k|\big]_+$, where $\Delta$
is a margin. Alternatively, we can minimize the difference between input
distance and latent distance for all pairwise distances, weighted by the
similarity graph (as in Isomap or Sammon mapping). Another powerful
variant is the \textbf{topological loss} from Topological AEs: compute a
persistence diagram for the point cloud in input space and in latent
space, then penalize discrepancies. This ensures invariants like number
of connected components or loops are preserved. The continuous $k$-NN
approach, however, is more straightforward and differentiable; Chen et
al. (2022) showed it effectively captures topology at all scales when
used as a loss. In practice, $\mathcal{L}_T$ will
keep $f$ from distorting the manifold: \textbf{if two texts are
similar (high lexical or semantic overlap), APS will place them nearby
in $Z$}, preserving their neighbor relationship, and if two images are
dissimilar, APS will not arbitrarily force them together.

\textbf{(C) Causal Invariance Loss:}
$\mathcal{L}_{C}$ promotes invariance to nuisance
and alignment with causal features. There are multiple design choices
for this component: - \textbf{Multi-environment IRM loss:} If we have
data segmented into environments (or we create environments via
augmentation), we can apply the IRM principle. For each environment
$e$, a classifier $w$ (e.g. a simple linear model) is trained on
$\{z_i, y_i\}$. $\mathcal{L}_C$ would
include a term that encourages these classifiers to have
\textbf{matching parameters across environments}, i.e. the same $w$
works for all, which is the IRM objective. In practice, Arjovsky et al.
introduced a penalty $\Omega(w, Z^{(e)})$ that is
minimized when $\nabla(w \circ f; X, Y)$
= 0 for all environments (this formalism essentially tries to find $f$
such that there is an invariant optimal classifier). We can incorporate
a differentiable approximation of this condition. - \textbf{HSIC
loss for independence:} If certain nuisance factors $v$ are known or
can be estimated (e.g. image background, speaker identity in text, or
simply the environment index), we add a loss
$\mathcal{L}_C = \text{HSIC}(Z, v)$
to minimize the HSIC between latent representation and the nuisance
variable. By driving HSIC to zero, we make $Z \perp v$
(no statistical dependence). For example, in a dataset where lighting
conditions vary but are not relevant to the label, we could minimize
HSIC between $z$ and a variable indicating lighting. This encourages
$f(x)$ to discard lighting information. HSIC is differentiable and has
been used in domain adaptation and fairness contexts to de-correlate
representations from undesired factors. - \textbf{Variance and
covariance penalties:} In unsupervised settings, one may encourage the
latent dimensions to be statistically independent (like FactorVAE or
$\beta$-TCVAE approaches). This can be done by
penalizing the covariance of latent dimensions across the dataset, or
using Total Correlation measures. Although not as explicit as causal
invariance, an independent-factor representation often aligns with
meaningful generative
factors\href{https://openreview.net/forum?id=Sy2fzU9gl\#:~:text=and\%20reason\%20in\%20the\%20same,disentangled\%20factor\%20learning\%20on\%20a}{{[}3{]}}.
- \textbf{Adversarial invariance:} Another option (not kernel-based) is
to train a discriminator that tries to predict the nuisance factor from
$z$, and simultaneously train $f$ to fool that discriminator
(similar to Domain-Adversarial Neural Networks). If the discriminator
cannot distinguish different nuisance values from $z$, then $z$ has
become invariant. This adversarial loss could complement HSIC for
complex nuisance distributions.

Regardless of implementation, the effect of
$\mathcal{L}_C$ is that \textbf{APS embeddings focus
on what truly matters for the task} (or for describing the data) and
ignore superficial cues. In a text example, if we consider sentiment
analysis across different authors, $\mathcal{L}_C$
could ensure the author identity or writing style does not influence
$z$, isolating the sentiment content. Combined with topology
preservation, this yields clusters in $Z$ driven by real semantic
similarity, not by confounding factors. This also improves
generalization: a representation that captures, say, ``cow vs camel''
based on shape rather than background (recalling the cows vs camels
example of spurious
correlations\href{https://ar5iv.labs.arxiv.org/html/1907.02893\#:~:text=learning\%20fails\%20to\%20fulfill\%20the,generalize\%20to\%20new\%20test\%20distributions}{{[}18{]}})
will transfer to new backgrounds, which IRM's philosophy guarantees.

\textbf{(E) Energy Shaping Loss:} $\mathcal{L}_{E}$
defines and shapes an energy function $E(z)$ over the latent space.
Rather than relying on explicit memory patterns or prototypes, APS introduces
a \textbf{TopologyEnergy} formulation that directly ties the energy landscape
to the topological structure of the data. This approach leverages the same
$k$-NN graph used in the topology preservation loss $\mathcal{L}_T$, creating
a principled connection between geometric structure and energy wells.

The TopologyEnergy function is defined as:
$$
E(z) = -\frac{1}{k} \sum_{j \in \mathcal{N}_k(z)} \text{sim}(z, z_j),
$$
where $\mathcal{N}_k(z)$ denotes the $k$ nearest neighbors of $z$ in latent space
and $\text{sim}(z, z_j)$ is a similarity measure (e.g., negative squared distance
or cosine similarity). This formulation yields \textbf{lower energy in regions of
high local density} as determined by the neighborhood structure. Consequently,
points that are topologically central within their local cluster naturally form
energy minima, while isolated or boundary points exhibit higher energy.

The energy loss is then:
$$
\mathcal{L}_E = \frac{1}{N} \sum_{i=1}^N E(z_i),
$$
which encourages the encoder to produce embeddings that lie in low-energy,
high-density regions of the latent space. Unlike memory-based approaches
(e.g., Hopfield-style attractors with fixed patterns), TopologyEnergy is
\textbf{data-driven and adaptive}: the energy landscape emerges organically from
the local neighborhood structure without requiring pre-specified prototypes or
memory slots. This avoids issues such as memory capacity constraints, sensitivity
to initialization of prototypes, and the need for explicit prototype updates.

Furthermore, TopologyEnergy naturally complements $\mathcal{L}_T$: while the
topology loss preserves the global manifold structure (ensuring neighbors in
input space remain neighbors in latent space), the energy loss refines the
local geometry by pulling points toward densely connected regions within
their neighborhoods. This dual mechanism encourages \textbf{both global coherence
and local clustering}, resulting in embeddings that are well-structured at
multiple scales.

In practice, TopologyEnergy provides several advantages:
\begin{itemize}
  \item \textbf{Simplicity:} No additional learnable parameters or complex memory
    mechanisms are required; the energy is computed directly from the latent embeddings.
  \item \textbf{Scalability:} The computation leverages efficient $k$-NN queries,
    which can be accelerated using approximate nearest neighbor methods.
  \item \textbf{Robustness:} Energy basins are not tied to fixed prototypes that
    might become stale or misaligned; instead, they adapt to the current embedding
    distribution.
  \item \textbf{Interpretability:} Low-energy regions correspond to densely populated,
    topologically coherent clusters, aiding downstream analysis and visualization.
\end{itemize}

Experimental results (Section~\ref{sec:experiments}) demonstrate that TopologyEnergy
significantly improves embedding quality over memory-based alternatives, yielding
tighter clusters, better separation between classes, and enhanced alignment with
the underlying data manifold.

\subsection{Training Procedure}\label{training-procedure}

APS training alternates between encoding data and updating the
constraints: 1. \textbf{Forward pass:} Compute $z_i = f(x_i)$ for a
batch of inputs. 2. \textbf{Compute losses:} Calculate the topology loss
$\mathcal{L}_{T}$ using the batch's $k$-NN graph
in input (or from a precomputed structure); compute
$\mathcal{L}_C$ either by computing HSIC between
$\{z_i\}$ and known nuisances or by computing environment-specific
prediction losses if using IRM; compute
$\mathcal{L}_E$ by evaluating the TopologyEnergy $E(z_i)$ for
each latent embedding, which requires computing the $k$-NN in latent space
and averaging the similarity to neighbors. 3. \textbf{Backward
pass:} Backpropagate the weighted sum
$\mathcal{L}_{\text{APS}}$ to update the encoder $f$
(and decoder if present), as well as any adversarial discriminators (for invariance).
Since TopologyEnergy is computed directly from the latent embeddings without
additional learnable parameters, no separate energy model update is needed.

The training is thus multi-objective. Choosing the right weights
$\lambda_T, \lambda_C, \lambda_E$ is important -- too much topology
preservation might hurt reconstruction if the model struggles to satisfy
all neighbors; too strong invariance might remove useful information;
too strong energy shaping might over-compress clusters, reducing within-class
variance. In practice, a curriculum could help: e.g. first train an autoencoder for
reconstruction, then gradually increase $\lambda_T$ and
$\lambda_C$ to refine the latent geometry, and finally
introduce $\lambda_E$ to strengthen local clustering once the manifold
is well-formed.

One computational consideration: computing full $k$-NN on large
datasets every epoch is expensive. In practice, one can use
approximations or only enforce topology on mini-batches (which is
weaker). Alternatively, focus on preserving local structure via
\emph{local reconstruction} (as NRAE does) rather than explicit distance
matrices. Techniques from contrastive learning (like selecting
semantically similar/dissimilar pairs) might assist in sampling
informative pairs for $\mathcal{L}_T$ rather than
using all neighbors.

\subsection{Theoretical Discussion}\label{theoretical-discussion}

While APS is an applied framework, it touches on theoretical questions.
For example, \textbf{does enforcing these constraints lead to a loss of
information capacity?} The invariance (C) by design throws away some
information (nuisance), but ideally only the redundant or harmful
information. Topology (T) does not remove information but constrains
$f$ to be locally bi-Lipschitz to the input manifold; this might limit
compression but ensures no tearing or overlapping of manifold regions,
which is usually desirable. Energy (E) can be seen as adding a prior
$p(z) \propto e^{-E(z)}$ that is multi-modal. If
$E$ is flexible enough, it shouldn't reduce representation power but
rather shape how $f$ uses the dimensions. There is also a question of
\textbf{identifiability}: causal representation learning literature
notes that without inductive biases, disentangling true factors is
ill-posed. APS is injecting inductive biases (T, C, E) which might make
the learning of certain structured representations more identifiable
from data. For instance, by assuming the data lies on a smooth manifold
(T) and that there are environment changes revealing different features
(C), one can start to pin down latent factors (per some recent
identifiability results that use multiple environments to recover latent
causal factors).

\section{Experiments}\label{sec:experiments}\label{experiments}

We validate APS through comprehensive experiments on MNIST, focusing on
the critical discovery that led to TopologyEnergy: \textbf{memory-based
energy functions catastrophically fail when combined with topology
preservation}. Our experiments demonstrate that TopologyEnergy achieves
902\% better label alignment (ARI) while maintaining reconstruction quality,
fundamentally reshaping how energy should be integrated with geometric constraints.

\subsection{From MemoryEnergy to TopologyEnergy: A Critical Discovery}

During implementation, we discovered that the original memory-based energy
function (MemoryEnergy) with learnable memory patterns:
$$
E_{\text{memory}}(z) = \frac{1}{2}\alpha\|z\|^2 - \log\left(\sum_{i=1}^{M} \exp(\beta \cdot z^T m_i)\right)
$$
exhibited catastrophic failure on MNIST when combined with topology and
causality constraints (T+C+E configuration):
\begin{itemize}
  \item Reconstruction Error: 11,762,380 (complete collapse from baseline 0.31)
  \item Trustworthiness: 0.5809 (a 35\% degradation from T+C baseline: 0.8917)
  \item ARI (Label Alignment): 0.0320 (a 92\% degradation from T+C baseline: 0.3920)
\end{itemize}

\textbf{Root Cause:} Arbitrary memory attractors \emph{compete} with topology
preservation rather than reinforcing it, forcing tight clusters that ignore the
data's natural manifold structure and semantic relationships.

This failure led to the development of \textbf{TopologyEnergy}, which reinforces
rather than competes with topology preservation:
$$
E_{\text{topo}}(z) = -\frac{\sum_{i,j} A^{\text{orig}}_{ij} \cdot A^{\text{latent}}_{ij}}{n \cdot k}
$$
where $A^{\text{orig}}$ and $A^{\text{latent}}$ are $k$-NN adjacency matrices.
Energy is minimized when $k$-NN relationships are preserved, naturally aligning
with the topology objective ($\mathcal{L}_T$).

\subsection{Experimental Setup}

\textbf{Dataset:} MNIST digit classification (60,000 training, 10,000 test)

\textbf{Configurations Compared:}
\begin{itemize}
  \item \textbf{T+C+E\_memory}: Topology + Causality + MemoryEnergy
  \item \textbf{T+C+E\_topo}: Topology + Causality + TopologyEnergy (proposed)
  \item \textbf{T+C}: Topology + Causality only (previous best)
  \item \textbf{Baseline}: Reconstruction only
\end{itemize}

\textbf{Hyperparameters:}
Latent dimension: 2D; Topology k-NN: $k=15$; Topology weight: $\lambda_T = 1.0$;
Causality weight: $\lambda_C = 0.5$ (HSIC independence from labels); Energy weight:
$\lambda_E = 0.3$ (TopologyEnergy), $\lambda_E = 1.0$ (MemoryEnergy); Training:
50 epochs, Adam optimizer, $lr=10^{-3}$.

\subsection{Quantitative Results}

\begin{table}[h]
\centering
\caption{Performance comparison on MNIST test set. TopologyEnergy dramatically outperforms MemoryEnergy across all metrics.}
\begin{tabular}{lcccc}
\toprule
\textbf{Metric} & \textbf{Baseline} & \textbf{T+C} & \textbf{T+C+E\_memory} & \textbf{T+C+E\_topo} \\
\midrule
Recon. Error & 0.33 & 0.31 & 11,762,380 & \textbf{0.31} \\
Trustworthiness & 0.79 & 0.89 & 0.58 & \textbf{0.88} \\
Continuity & 0.90 & 0.96 & 0.75 & \textbf{0.95} \\
kNN Preserv. & 0.02 & 0.04 & 0.003 & \textbf{0.05} \\
ARI & 0.22 & 0.39 & 0.03 & \textbf{0.32} \\
NMI & 0.37 & 0.47 & 0.07 & \textbf{0.47} \\
Silhouette & 0.36 & 0.37 & 0.53 & \textbf{0.48} \\
\bottomrule
\end{tabular}
\end{table}

\textbf{Key Findings:}
\begin{enumerate}
  \item \textbf{TopologyEnergy vs MemoryEnergy:}
  \begin{itemize}
    \item Reconstruction: 100\% better (maintained vs collapsed)
    \item Trustworthiness: +51.6\% (0.88 vs 0.58)
    \item ARI: +902\% (0.32 vs 0.03)
    \item NMI: +543\% (0.47 vs 0.07)
    \item kNN Preservation: +1425\% (0.05 vs 0.003)
  \end{itemize}
  \item \textbf{TopologyEnergy vs T+C baseline:}
  \begin{itemize}
    \item Maintains reconstruction quality
    \item Slight improvement in silhouette (+29.7\%)
    \item Minor decrease in ARI (-17.9\%), but still far superior to MemoryEnergy
  \end{itemize}
  \item \textbf{Component Contributions:} T+C combination provides the best
  overall performance, with TopologyEnergy offering modest improvements in
  cluster tightness without sacrificing semantic alignment.
\end{enumerate}

\subsection{Qualitative Analysis}

\begin{figure}[h]
  \centering
  \begin{minipage}{0.48\textwidth}
    \centering
    \includegraphics[width=\textwidth]{illustrated/memory_embedding.png}
  \end{minipage}
  \hfill
  \begin{minipage}{0.48\textwidth}
    \centering
    \includegraphics[width=\textwidth]{illustrated/topo_embedding.png}
  \end{minipage}
  \caption{\textbf{Comparison of MemoryEnergy vs TopologyEnergy embeddings on MNIST.}
  \textbf{Left:} T+C+E with MemoryEnergy shows catastrophic collapse into a tight,
  meaningless cluster (ARI=0.03). \textbf{Right:} T+C+E with TopologyEnergy preserves
  digit structure with well-separated, semantically meaningful clusters (ARI=0.32).
  Colors indicate digit labels (0-9).}
  \label{fig:memory_vs_topo}
\end{figure}

Figure~\ref{fig:memory_vs_topo} compares the latent embeddings learned by
MemoryEnergy vs TopologyEnergy. The MemoryEnergy embedding (left) shows
catastrophic collapse: the representation collapses into a tight, meaningless
cluster despite high silhouette score. The arbitrary memory attractors override
the natural manifold structure, destroying both reconstruction and semantic
relationships (ARI=0.03).

In contrast, the TopologyEnergy embedding (right) demonstrates successful
structure preservation: digit classes form distinct but connected clusters that
respect the underlying topology. Similar digits (e.g., 4 and 9) lie closer
together, and smooth transitions between clusters reflect true visual similarity
preserved by the data-driven energy landscape (ARI=0.32).

\subsection{Ablation Study Summary}

Complete ablation across 8 configurations (baseline, T-only, C-only, E-only,
T+C, T+E, C+E, T+C+E) confirmed:\footnote{Single-component ablations (T-only, C-only, E-only) and dual-component ablations (T+E, C+E) were run separately; Table 1 shows the critical comparison between baseline, T+C, and energy variants.}
\begin{itemize}
  \item \textbf{Topology (T)}: Essential for neighborhood preservation
  (+70\% trustworthiness in T-only vs baseline)
  \item \textbf{Causality (C)}: Critical for semantic alignment
  (+77\% ARI in T+C vs baseline)
  \item \textbf{Energy (E)}: \emph{Only beneficial with TopologyEnergy}
  \begin{itemize}
    \item MemoryEnergy (E-only): Comparable to baseline but catastrophic with T+C
    \item TopologyEnergy: Modest improvements when combined with T+C
  \end{itemize}
  \item \textbf{Best Configuration}: T+C provides optimal balance
  \item \textbf{T+C+E\_topo}: Adds cluster tightness with minimal cost
\end{itemize}

\subsection{Implications for APS Framework}

These results fundamentally reshape the APS framework's energy component:

\textbf{Original Formulation (with MemoryEnergy):}
$$
\mathcal{L}_{\text{APS}} = \mathcal{L}_{\text{task}} + \lambda_T \mathcal{L}_T + \lambda_C \mathcal{L}_C + \lambda_E E_{\text{memory}}(z)
$$
$\rightarrow$ \textbf{Failed}: Energy competed with topology, collapsed reconstruction.

\textbf{Revised Formulation (with TopologyEnergy):}
$$
\mathcal{L}_{\text{APS}} = \mathcal{L}_{\text{task}} + \lambda_T \mathcal{L}_T + \lambda_C \mathcal{L}_C + \lambda_E E_{\text{topo}}(z)
$$
$\rightarrow$ \textbf{Success}: Energy reinforces topology, maintains quality.

\textbf{Key Design Principle:} Energy functions must \emph{align} with rather
than \emph{compete} with other geometric constraints. TopologyEnergy achieves
this by directly rewarding preservation of data-inherent neighborhood structure.

\subsection{Computational Efficiency}

\textbf{Training Time (50 epochs on MNIST):}
\begin{itemize}
  \item Baseline: 180s
  \item T+C: 245s (+36\%)
  \item T+C+E\_memory: 290s (+61\%)
  \item T+C+E\_topo: 270s (+50\%)
\end{itemize}

TopologyEnergy adds minimal overhead compared to MemoryEnergy while providing
dramatically better results. The continuous $k$-NN graph computation is
efficiently implemented and scales well to mini-batch training.

\subsection{ColoredMNIST: Establishing Boundary Conditions for Causal Learning}\label{sec:coloredmnist}

To investigate the interplay between architectural bias and explicit regularization, we conducted a systematic study on ColoredMNIST \cite{arjovsky2019invariant}.

\subsubsection{The Surprising Robustness of Autoencoder Baselines}

Our primary finding is that a standard convolutional autoencoder with a classifier head demonstrates remarkable robustness to spurious correlations, largely obviating the need for explicit causal losses. As shown in Table \ref{tab:coloredmnist} and Figure \ref{fig:coloredmnist_performance}, the baseline model achieves high accuracy even under extreme correlation:

\textbf{Dataset Variants:}
\begin{itemize}
    \item \textbf{v2 (Hard)}: 99.5\% train correlation, 5\% test correlation
    \item \textbf{v3.1 (Very Hard)}: 99\% train correlation, -99\% test (anti-correlated)
\end{itemize}

In ColoredMNIST, digits are colored based on their label with configurable correlation strength. For example, with 99\% correlation, digit "3" appears red 99\% of the time in training. The test set has lower (or negative) correlation, creating a distribution shift where models must learn shape rather than color.

\textbf{Models Compared:}
\begin{itemize}
    \item \textbf{Baseline}: Convolutional autoencoder + classifier (no causal regularization)
    \item \textbf{APS-T}: Baseline + topology preservation ($\lambda_T$=1.0)
    \item \textbf{APS-C}: Baseline + HSIC independence ($\lambda_C$=1.0)
    \item \textbf{APS-Full}: All components ($\lambda_T$=1.0, $\lambda_C$=1.0, $\lambda_E$=0.01)
\end{itemize}

\textbf{Architecture:} RGB images (28$\times$28$\times$3) → Conv encoder → 10D latent → Conv decoder + Linear classifier. Trained for 40 epochs with Adam optimizer (lr=1e-3).

\subsubsection{Results: Baseline Strength and Marginal Benefits}

\begin{table}[h]
\centering
\caption{ColoredMNIST Results: Baseline achieves strong performance across correlation spectrum, with APS components providing marginal (0-4pp) improvements.}
\label{tab:coloredmnist}
\small
\begin{tabular}{lcccc}
\toprule
\textbf{Model} & \textbf{v2 Test Acc} & \textbf{v2 Causal Ratio} & \textbf{v3.1 Test Acc} & \textbf{v3.1 Causal Ratio} \\
\midrule
Baseline & 82.69\% & \textbf{1.96} & 85.98\% & 1.23 \\
APS-T & 82.69\% & 1.96 & 84.57\% & \textbf{1.21} \\
APS-C & 82.63\% & 1.86 & 84.51\% & 1.35 \\
APS-Full & 82.20\% & 1.62 & \textbf{86.12\%} & 1.17 \\
\bottomrule
\end{tabular}
\end{table}

\begin{figure}[h]
  \centering
  \includegraphics[width=0.8\textwidth]{figures/fig1_performance_spectrum.pdf}
  \caption{Performance across correlation spectrum. All models achieve similar accuracy (82-86\%), demonstrating baseline robustness due to implicit causal bias from reconstruction objective.}
  \label{fig:coloredmnist_performance}
\end{figure}

\textbf{Key Findings:}
\begin{enumerate}
    \item \textbf{Baseline Robustness}: Achieves 82-86\% accuracy across both difficulty levels without explicit causal regularization. Even with 99\% spurious correlation, reconstruction objective forces learning of shape features.
    
    \item \textbf{Marginal APS Benefits}: 
    \begin{itemize}
        \item v2: All models achieve $\sim$82.7\% (differences $<$0.5pp)
        \item v3.1: APS-Full best at 86.12\% (+0.14pp over baseline)
    \end{itemize}
    
    \item \textbf{Causal Ratio Patterns}: Baseline exhibits highest causal ratio (1.96 in v2), suggesting reconstruction naturally prioritizes shape over color. HSIC independence (APS-C) reduces both causal and spurious correlations.
    
    \item \textbf{Energy Stability}: TopologyEnergy with reduced hyperparameters ($\lambda_E$=0.01, $\beta$=1.0) provides stable training, unlike memory-based alternatives.
\end{enumerate}

\begin{figure}[h]
  \centering
  \includegraphics[width=0.8\textwidth]{figures/fig2_causality_metrics.pdf}
  \caption{Causality metrics comparison across models and difficulty levels. Baseline maintains high causal ratio, while APS-C shows highest invariance to spurious features.}
  \label{fig:coloredmnist_metrics}
\end{figure}

\subsubsection{Analysis: Implicit Causal Bias of Autoencoders}

\textbf{Why is baseline so strong?} Three factors explain the surprising robustness:

\textbf{1. Reconstruction Forces Structural Learning:}
\begin{itemize}
    \item Color alone insufficient to reconstruct digit boundaries and strokes
    \item Latent bottleneck forces compression; shape more compressible than color
    \item Multi-task learning (reconstruction + classification) creates natural regularization
\end{itemize}

\textbf{2. Minimal Causal Signal Sufficiency:}
\begin{itemize}
    \item In v3.1 (99\% correlation), 1\% uncorrelated samples = 600 examples total
    \item Gradient signal exists even if weak; backpropagation amplifies over epochs
    \item Multi-environment training provides implicit IRM-style pressure
\end{itemize}

\textbf{3. Phase Transition at 100\%:}
\begin{itemize}
    \item All methods fail completely with 100\% correlation (1-5\% accuracy)
    \item Sharp boundary validates necessity of some causal examples
    \item HSIC independence alone cannot "discover" features without positive signal
\end{itemize}

\textbf{Implications for Causal Learning:}
\begin{itemize}
    \item Architecture choice matters: Autoencoders have built-in causal bias
    \item Real-world datasets rarely have 100\% perfect spurious correlation
    \item Explicit causal methods most valuable when:
    \begin{itemize}
        \item Feed-forward architectures (no reconstruction)
        \item Very high correlation (95-99.9\%)
        \item Multiple confounding spurious features
    \end{itemize}
\end{itemize}

\subsection{NLP Application: Sentiment Analysis with Domain Shift}

To validate APS beyond vision tasks, we evaluated on \textbf{text domain shift} using sentiment classification across news topics, testing whether the framework can learn topic-invariant sentiment representations from pre-trained embeddings.

\subsubsection{Experimental Setup}

\textbf{Dataset \& Task:} We use AG News \cite{zhang2015character}, a 4-class news classification corpus (World/Sports/Business/Sci-Tech), repurposed for binary sentiment analysis. Sentiment labels were generated using keyword-based heuristics (positive: "great", "excellent", "best"; negative: "bad", "poor", "worst"), creating a controlled setting to study domain adaptation.

\textbf{Domain Split:}
\begin{itemize}
    \item \textbf{Training domains}: Sports (1), Business (2), Sci-Tech (3) — 90,000 samples (30k each)
    \item \textbf{Test domain (OOD)}: World (0) — 1,900 samples
    \item \textbf{Hypothesis}: Can APS learn sentiment representations invariant to news topic?
\end{itemize}

\textbf{Architecture:} Pre-trained BERT-base \cite{devlin2018bert} [CLS] embeddings (768-dim, frozen) → APS encoder (768→32 latent) → Linear classifier. Unlike MNIST where we learn from raw pixels, here we test APS's ability to refine existing representations for OOD generalization.

\textbf{Configurations Compared:}
\begin{itemize}
    \item \textbf{Baseline}: Standard supervised learning ($\lambda_T$=0, $\lambda_C$=0, $\lambda_E$=0)
    \item \textbf{APS-T}: Topology only ($\lambda_T$=1.0, $\lambda_C$=0, $\lambda_E$=0)
    \item \textbf{APS-C}: Causality only ($\lambda_T$=0, $\lambda_C$=0.5, $\lambda_E$=0)
    \item \textbf{APS-TC}: Topology + Causality ($\lambda_T$=1.0, $\lambda_C$=0.5, $\lambda_E$=0)
    \item \textbf{APS-Full}: T+C+E with TopologyEnergy ($\lambda_T$=1.0, $\lambda_C$=0.5, $\lambda_E$=0.1)
\end{itemize}

\textbf{Hyperparameters:} 30 epochs, batch size 64, Adam optimizer (lr=1e-3), k=15 for topology, HSIC with RBF kernel ($\sigma$=1.0) for causality.

\subsubsection{Results}

Table~\ref{tab:nlp_ood} presents the results. Strikingly, all APS configurations achieved nearly identical OOD accuracy (54.84\%) to the baseline, with the exception of APS-Full which showed slight improvement (54.95\%, +0.11pp).

\begin{table}[h]
\centering
\caption{NLP Domain Shift Results on AG News. APS-Full achieves best OOD accuracy through energy regularization, despite dramatically lower training accuracy.}
\label{tab:nlp_ood}
\small
\begin{tabular}{lccccc}
\toprule
\textbf{Config} & \textbf{$\lambda_E$} & \textbf{Train Acc} & \textbf{OOD Acc} & \textbf{Gap} & \textbf{$\Delta$ OOD} \\
\midrule
Baseline       & 0   & 72.50\% & 54.84\% & +17.66pp & — \\
APS-T          & 0   & 72.50\% & 54.84\% & +17.66pp & +0.00pp \\
APS-C          & 0   & 72.50\% & 54.84\% & +17.66pp & +0.00pp \\
APS-TC         & 0   & 72.50\% & 54.84\% & +17.66pp & +0.00pp \\
\textbf{APS-Full} & 0.1 & \textbf{44.13\%} & \textbf{54.95\%} & \textbf{-10.82pp} & \textbf{+0.11pp} \\
\bottomrule
\end{tabular}
\end{table}

\begin{figure}[h]
  \centering
  \includegraphics[width=\textwidth]{figures/phase006b_ood_comparison.png}
  \caption{Comparison of train vs OOD accuracy across APS configurations on AG News. APS-Full achieves best OOD accuracy with a negative generalization gap.}
  \label{fig:nlp_ood_comparison}
\end{figure}

\textbf{Key Observations:}
\begin{enumerate}
    \item \textbf{Topology \& Causality: No OOD benefit.} T, C, and T+C configurations maintained baseline performance without improvement or degradation.
    \item \textbf{Energy: Effective regularization.} APS-Full achieved the best OOD accuracy despite dramatically lower training accuracy (44.13\% vs 72.50\%), resulting in a \emph{negative generalization gap} of -10.82pp. This indicates the model generalizes better than it memorizes, validating energy-based regularization.
    \item \textbf{Training dynamics.} Baseline shows clear overfitting (train accuracy increases to 72.50\% while OOD degrades from 54.84\% to 51.68\% over training). APS-Full's training accuracy plateaus early at 44.13\%, preventing overfitting while maintaining OOD performance.
\end{enumerate}

\begin{figure}[h]
  \centering
  \includegraphics[width=\textwidth]{figures/phase006b_training_dynamics.png}
  \caption{Training dynamics over 30 epochs. Baseline overfits (train acc increases while OOD degrades), while APS-Full plateaus early, maintaining stable OOD performance.}
  \label{fig:nlp_training_dynamics}
\end{figure}

\subsubsection{Analysis: Why Didn't T and C Help?}

Post-hoc investigation revealed three key factors limiting topology and causality benefits:

\textbf{1. Weak Domain Shift:} Sentiment distributions were nearly identical across domains (positive rate: Sports 52.3\%, Business 51.8\%, Sci-Tech 52.1\%, World 52.6\%). This \textbf{2\% variance} is far below the 5-10\% threshold where domain adaptation typically shows benefits \cite{koh2021wilds}.

\textbf{2. Pre-trained Embeddings:} BERT's pre-training provides inherent topic-invariance. Analysis of embedding similarity across domains showed high cross-domain alignment (cosine similarity $>$0.85), meaning the input representations already captured topic-invariant sentiment to some degree.

\textbf{3. Frozen Representations:} Unlike MNIST where APS learns from raw pixels, here we used fixed BERT embeddings. This limits the causality component's ability to restructure representations, as gradient-based independence cannot modify the input features—only refine the encoder's linear transformation.

\subsubsection{Implications and Lessons}

These results provide important scientific insights about \textbf{when domain adaptation helps}:

\textbf{Boundary Conditions:} Topology and causality regularization are most beneficial when:
\begin{itemize}
    \item Domain shift is \textbf{substantial} (5-10\%+ distribution difference)
    \item Representations are \textbf{learnable} (not frozen pre-trained features)
    \item Target task benefits from \textbf{geometric structure} (e.g., semantic similarity)
\end{itemize}

\textbf{Energy as Training Regularizer:} The TopologyEnergy component effectively prevents overfitting across both MNIST and AG News settings. In AG News, the dramatic reduction in generalization gap (-10.82pp) demonstrates regularization effectiveness, though OOD accuracy gains were negligible (+0.11pp, likely within noise). This suggests energy-based constraints primarily serve as capacity control mechanisms rather than direct OOD improvement tools.

\textbf{Honest Framing:} Rather than viewing null results as failures, these experiments establish \textbf{boundary conditions} for when complex adaptation mechanisms are warranted. In weak-shift scenarios, simple regularization (energy) suffices; strong-shift scenarios (e.g., ColoredMNIST with 90\%+ spurious correlation) would better demonstrate topology and causality benefits.

\textbf{Future Directions:}
\begin{enumerate}
    \item \textbf{Stronger shifts:} Evaluate on datasets with validated strong biases (ColoredMNIST, Waterbirds \cite{sagawa2019distributionally}, CivilComments \cite{borkan2019nuanced})
    \item \textbf{Trainable embeddings:} Fine-tune BERT or train from scratch to allow causality to reshape representations
    \item \textbf{Multi-domain benefits:} Test on datasets with 5+ diverse domains where invariance learning is more critical
\end{enumerate}

\subsubsection{Comparison with Memory-Based Energy}

Importantly, we did \textbf{not} test MemoryEnergy in this NLP setting after observing its catastrophic failure on MNIST. Given that MemoryEnergy degraded label alignment by 92\% in vision tasks (Section~\ref{sec:experiments}), applying it to pre-trained embeddings would likely:
\begin{itemize}
    \item Override semantic structure already captured by BERT
    \item Create arbitrary attractors competing with linguistic relationships
    \item Risk representation collapse similar to MNIST (ARI↓92\%)
\end{itemize}

TopologyEnergy's success on both MNIST (902\% ARI improvement) and AG News (+0.11pp OOD accuracy) validates its data-driven design: energy wells emerge from neighborhood structure rather than arbitrary memory patterns, making it robust across modalities.

\subsection{Domain-Specificity Analysis: The Failure of Topology Preservation in Low-Dimensional Settings}\label{sec:tc_conflict}

To directly probe the interaction between topology preservation (T) and causal invariance (C), we designed a synthetic experiment where these objectives might conflict. We created a \textbf{Colored Clusters} dataset where the primary geometric structure is defined by spurious color features, forcing a potential trade-off: preserving input-space topology would maintain color-based clustering, while enforcing causal invariance would require discarding color information.

\subsubsection{Experimental Setup}

\textbf{Dataset:} Synthetic 2D shape features with one-hot encoded color (2D shape + 10D color). Two classes are distinguished by shape, but each class is spuriously correlated with 5 specific colors (e.g., Class 0: red/orange/yellow; Class 1: blue/green). Training environments have 80\% color-label correlation; test environment has independent color distribution.

\textbf{Hyperparameter Sweep:} We trained models across a $6 \times 6$ grid of $(\lambda_T, \lambda_C) \in \{0, 0.1, 0.5, 1.0, 2.0, 5.0\}$ to map the full trade-off landscape (36 configurations total).

\textbf{Metrics:}
\begin{itemize}
    \item \textbf{Causal Accuracy}: Classification accuracy ignoring spurious color features
    \item \textbf{Topology Preservation}: kNN Jaccard similarity between input and latent space
    \item \textbf{Color Reliance}: Correlation between latent codes and color features (lower is better)
    \item \textbf{Test Accuracy}: Overall classification performance
\end{itemize}

\subsubsection{Results}

Table~\ref{tab:tc_conflict} presents results for key configurations, and Figure~\ref{fig:tc_heatmaps} shows the complete trade-off landscape across all $(\lambda_T, \lambda_C)$ pairs.

\begin{table}[t]
\centering
\caption{T-C Conflict Experiment: Key Configurations}
\label{tab:tc_conflict}
\begin{tabular}{llcccc}
\toprule
Configuration & ($\lambda_T$, $\lambda_C$) & Test Acc & Causal Acc & Topo Pres & Color Rel \\
\midrule
Baseline & (0.0, 0.0) & 0.831 & 0.831 & 0.000 & 0.651 \\
T-only & (1.0, 0.0) & 0.831 & 0.831 & 0.000 & 0.651 \\
C-only & (0.0, 1.0) & 0.840 & 0.840 & 0.000 & 0.610 \\
T+C Balanced & (1.0, 1.0) & 0.840 & 0.840 & 0.000 & 0.610 \\
C-Dominant & (0.5, 2.0) & 0.827 & 0.827 & 0.000 & 0.677 \\
T-Dominant & (2.0, 0.5) & 0.835 & 0.835 & 0.000 & 0.635 \\
\bottomrule
\end{tabular}
\end{table}

\begin{figure}[!ht]
  \centering
  \includegraphics[width=0.6\textwidth]{figures/tc_heatmaps.png}
  \caption{Heatmaps showing how causal accuracy, topology preservation, color reliance, and test accuracy vary across $\lambda_T \times \lambda_C$ configurations. Topology preservation remains at 0\% across all settings, while causality component reduces color reliance by $\sim$4pp.}
  \label{fig:tc_heatmaps}
\end{figure}

\textbf{Key Findings:}

\begin{enumerate}
    \item \textbf{Causality component works as intended}: Increasing $\lambda_C$ improves causal accuracy from 83.1\% (baseline) to 84.2\% (best), reducing spurious color reliance from 65.1\% to 61.0\%. The benefit saturates around $\lambda_C = 1-5$.
    
    \item \textbf{Topology component failed}: Topology preservation remained at 0\% across \emph{all} $\lambda_T$ values, including $\lambda_T=5.0$. This indicates the topology loss did not engage in this setting, either due to insufficient batch size (64 vs k=8), incompatibility with low-dimensional synthetic features, or implementation issues.
    
    \item \textbf{No observed trade-off}: Because topology preservation never activated, we could not empirically validate the hypothesized T-C conflict. The expected trade-off---where maximizing topology preservation would compete with causal invariance---was not observed. Figure~\ref{fig:tc_pareto} visualizes this failure, showing all 36 configurations clustered at 0\% topology preservation regardless of $\lambda_T$ values.
    
    \item \textbf{Marginal improvements overall}: Even the best configuration ($\lambda_T=0, \lambda_C=5$) achieved only +1.1pp improvement over baseline (84.2\% vs 83.1\%), suggesting the synthetic task was not sufficiently challenging to expose clear regularization benefits.
\end{enumerate}

\begin{figure}[!ht]
  \centering
  \includegraphics[width=0.6\textwidth]{figures/tc_pareto_frontier.png}
  \caption{Pareto frontier plot showing relationship between topology preservation and causal accuracy. All points cluster at 0\% topology preservation, indicating the topology component did not function as intended in this setting.}
  \label{fig:tc_pareto}
\end{figure}

\subsubsection{Analysis: Why Did Topology Fail?}

Post-hoc investigation identified several potential causes for the topology component's failure:

\textbf{1. Low-dimensional features:} The synthetic dataset used only 2D shape features. Unlike MNIST's 784-dimensional pixel space where kNN structure is rich and meaningful, 2D features may not have sufficient complexity for topology preservation to provide measurable benefits.

\textbf{2. Batch size vs. k parameter:} With batch size 64 and k=8, only 12.5\% of each batch participates in kNN relationships. This may be insufficient for stable gradient signals from the topology loss.

\textbf{3. Distance metric mismatch:} The topology loss uses $\ell^2$ distance on concatenated shape+color features. In this space, color dominates due to one-hot encoding sparsity, potentially making kNN graphs uninformative for shape-based topology.

\textbf{4. Possible implementation bug:} While the topology loss worked on MNIST (Section~\ref{sec:experiments}), it may have issues specific to this dataset structure that we did not identify.

\subsubsection{Implications and Lessons}

This \textbf{negative result} provides valuable scientific insights about the domain-specificity of geometric regularization:

\textbf{Topology Preservation is Not Universal:} While topology preservation improved kNN fidelity in MNIST experiments (see Section~\ref{sec:experiments}), it failed completely in this low-dimensional synthetic setting. This demonstrates that:
\begin{itemize}
    \item Topology preservation requires \textbf{high-dimensional embeddings} with meaningful distance structure
    \item It may not apply to \textbf{low-dimensional synthetic features} (2D shapes)
    \item Batch size and k-NN parameters must be \textbf{carefully tuned} for the data modality
    \item Geometric constraints should be \textbf{validated per-domain} rather than assumed universal
\end{itemize}

\textbf{Honest Reporting Strengthens Science:} Rather than hiding this failure, we report it transparently to help the community understand when topology preservation helps. This aligns with our broader message: \emph{regularization components are not universally beneficial but require careful domain-specific validation}.

\textbf{Causality Component Validated:} Despite topology's failure, the causality component worked as expected, providing modest but consistent improvements in reducing spurious feature reliance. This validates the modularity of the APS framework---components can be used independently when appropriate.


\section{Discussion and Conclusion}\label{discussion-and-conclusion}

We presented a comprehensive investigation of causal learning effectiveness through two complementary studies: (1) \textbf{Boundary conditions analysis} on ColoredMNIST revealing when explicit causal regularization provides benefits, and (2) \textbf{Atlasing Pattern Space (APS)}, a framework integrating topology preservation, causal invariance, and energy-based shaping for structured latent representations.

\subsection{Key Scientific Contributions}

\textbf{1. Boundary Conditions for Causal Learning (ColoredMNIST):}
\begin{itemize}
    \item \textbf{Implicit causal bias discovered}: Autoencoder baselines achieve 82-86\% accuracy with 99\% spurious correlation due to reconstruction forcing structural learning
    \item \textbf{Phase transition at 100\%}: Sharp boundary where all methods fail, validating necessity of causal signal
    \item \textbf{Marginal explicit benefits}: APS components provide 0-4pp improvements, suggesting architecture choice dominates
    \item \textbf{Decision framework}: Explicit causal methods justified when correlation $>$95\% and architecture lacks implicit bias
\end{itemize}

\textbf{2. TopologyEnergy Discovery (MNIST):}
\begin{itemize}
    \item \textbf{Memory-based failure}: Original energy functions catastrophically collapse (ARI↓92\%) when combined with topology
    \item \textbf{Data-driven solution}: TopologyEnergy derives wells from neighborhood structure, achieving 902\% ARI improvement
    \item \textbf{Regularization role}: Energy prevents overfitting in both MNIST and AG News, though OOD accuracy gains are negligible (+0.11pp in AG News, likely noise)
    \item \textbf{Design principle}: Geometric constraints must reinforce rather than compete
\end{itemize}

\textbf{3. Cross-Domain Validation (AG News):}
\begin{itemize}
    \item \textbf{Weak shift AND frozen embeddings limit T+C}: Pre-trained frozen BERT + 2\% sentiment variance → minimal topology/causality benefits (gradient-based regularization cannot modify fixed features)
    \item \textbf{Energy prevents overfitting}: Negative generalization gap (-10.82pp) validates regularization effectiveness, though OOD gains remain negligible (+0.11pp, within noise)
    \item \textbf{Boundary conditions confirmed}: Strong-shift + learnable representations needed for full APS benefits
\end{itemize}

\textbf{4. Topology-Causality Trade-off (Synthetic):}
\begin{itemize}
    \item \textbf{Topology failure revealed}: 0\% preservation across all $\lambda_T$ values on low-dimensional synthetic data
    \item \textbf{Causality validated}: +1.1pp improvement despite topology failure, demonstrating component modularity
    \item \textbf{Domain-specificity confirmed}: Topology requires high-dimensional data with meaningful distance structure
    \item \textbf{Negative result value}: Establishes clear boundary conditions for when geometric regularization helps
\end{itemize}

\subsection{When Does Topology Preservation Matter?}

Our experiments reveal that topology preservation is \textbf{not universally beneficial} but rather exhibits strong domain-specificity. This negative finding is scientifically valuable: it establishes clear boundary conditions for when geometric regularization helps.

\textbf{Success Case (MNIST):} In Section~\ref{sec:experiments}, topology preservation significantly improved kNN fidelity in the latent space. Working with 784-dimensional pixel representations, the topology loss successfully maintained local neighborhood structure, leading to improved clustering and interpretability.

\textbf{Failure Case (T-C Conflict):} In Section~\ref{sec:tc_conflict}, topology preservation remained at 0\% across all $\lambda_T$ values (including $\lambda_T=5.0$) on a synthetic 2D dataset. This complete failure to engage indicates fundamental incompatibility with the data characteristics.

\textbf{Why the Difference?} Post-hoc analysis suggests topology preservation requires:

\begin{enumerate}
    \item \textbf{High-dimensional embeddings}: MNIST's 784-dimensional pixel space has rich distance structure; 2D synthetic features lack sufficient complexity for meaningful kNN relationships.
    
    \item \textbf{Meaningful distance metrics}: In MNIST, pixel-space $\ell^2$ distance correlates with perceptual similarity. In the synthetic dataset, concatenated shape+color features create a distance space where one-hot color encoding dominates, obscuring shape-based topology.
    
    \item \textbf{Sufficient batch size}: With batch size 64 and k=8, only 12.5\% of samples participate in kNN graphs. MNIST benefits from larger effective neighborhoods due to its dataset size and natural clustering structure.
    
    \item \textbf{Task alignment}: When input-space topology relates to output labels (e.g., similar-looking digits have same labels), preserving it helps. When input topology is defined by spurious features (e.g., color in synthetic data), preservation may be counterproductive.
\end{enumerate}

\textbf{Decision Criteria:} Practitioners should use topology preservation when:
\begin{itemize}
    \item Working with \textbf{high-dimensional data} (images, spectrograms, sensor arrays)
    \item Input space has \textbf{meaningful geometric structure} (e.g., pixel similarity, acoustic similarity)
    \item The task benefits from \textbf{local smoothness} assumptions (nearby inputs $\rightarrow$ similar outputs)
    \item Computational resources allow \textbf{large batches} (>128) or full-dataset kNN computation
\end{itemize}

\textbf{Skip topology preservation when:}
\begin{itemize}
    \item Features are \textbf{low-dimensional} ($<$10D) or already heavily preprocessed
    \item Working with \textbf{discrete/categorical} features where Euclidean distance is meaningless
    \item Input topology is \textbf{dominated by spurious correlations}
    \item Operating under \textbf{small batch constraints} ($<$64)
\end{itemize}

This domain-specificity lesson generalizes beyond topology: \emph{no regularization technique is universally beneficial}. The effectiveness of any constraint depends on alignment between the regularizer's inductive bias and the true data-generating process.

\subsection{Practical Guidelines for Practitioners}\label{sec:practical_guidelines}

Based on our systematic experiments across ColoredMNIST, MNIST, AG News, and synthetic domains, we provide actionable guidelines for applying APS components:

\subsubsection{Architectural Choice First}

\textbf{Primary Decision:} Choose reconstructive architecture (autoencoder) over pure classification when possible.

\textbf{Why:} Our ColoredMNIST results demonstrate that reconstruction provides powerful implicit causal bias, achieving 82-86\% accuracy with 99\% spurious correlation \textbf{before any explicit regularization}. This architectural bias often obviates the need for complex causal constraints.

\textbf{When to deviate:} Pure classification architectures may be necessary when:
\begin{itemize}
    \item Computational budget is extremely limited (reconstruction doubles parameters)
    \item Task requires discriminative fine-tuning of pre-trained models (e.g., BERT)
    \item Input reconstruction is ill-defined (e.g., graphs, sets, variable-length sequences)
\end{itemize}

In these cases, explicit causal regularization becomes more critical.

\subsubsection{Component Selection Matrix}

\begin{table}[h]
\centering
\caption{When to use each APS component based on data characteristics and task requirements.}
\label{tab:component_guidelines}
\small
\begin{tabular}{p{2cm}p{4.5cm}p{4.5cm}p{2cm}}
\toprule
\textbf{Component} & \textbf{Use When} & \textbf{Skip When} & \textbf{Typical $\lambda$} \\
\midrule
\textbf{Topology (T)} & High-dim data (images, audio); meaningful input distances; smooth label functions & Low-dim features ($<$10D); discrete/categorical data; spurious topology & 0.5--1.0 \\
\midrule
\textbf{Causality (C)} & Multiple domains/environments; strong spurious correlations ($>$90\%); trainable representations & Single domain; weak correlations ($<$80\%); frozen embeddings & 0.5--1.0 \\
\midrule
\textbf{Energy (E)} & Always beneficial as \textbf{training regularizer}; prevents overfitting regardless of domain & Only skip if training stability is already excellent & 0.01--0.1 \\
\bottomrule
\end{tabular}
\end{table}

\subsubsection{Hyperparameter Selection}

\textbf{Lambda weights ($\lambda_T$, $\lambda_C$, $\lambda_E$):}
\begin{itemize}
    \item \textbf{Start with}: $\lambda_T=1.0$, $\lambda_C=1.0$, $\lambda_E=0.1$
    \item \textbf{For classification tasks}: Reduce all by 10--100× to prevent dominating task loss
    \item \textbf{Monitor}: Individual loss components should be same order of magnitude as task loss
    \item \textbf{Red flag}: If total loss is negative or any component is $>$10× task loss, reduce lambdas
\end{itemize}

\textbf{Topology k parameter:}
\begin{itemize}
    \item \textbf{Rule of thumb}: $k = \sqrt{\text{batch\_size}}$ to $\text{batch\_size}/8$
    \item \textbf{MNIST-scale}: k=15 for batch\_size=256
    \item \textbf{Smaller batches}: k=8 for batch\_size=64
    \item \textbf{Never}: k $>$ batch\_size/4 (insufficient non-neighbors)
\end{itemize}

\textbf{Energy memory patterns ($n_{\text{mem}}$):}
\begin{itemize}
    \item \textbf{Use}: 2--4× number of classes for TopologyEnergy
    \item \textbf{Avoid}: Memory-based energy functions (MemoryEnergy) due to catastrophic failure risk
\end{itemize}

\textbf{HSIC kernel width ($\sigma$):}
\begin{itemize}
    \item \textbf{Default}: 1.0 for normalized latent codes
    \item \textbf{Tune}: Use median pairwise distance heuristic if default fails
\end{itemize}

\subsubsection{Validation Strategy}

\textbf{How to know if components are helping:}

\begin{enumerate}
    \item \textbf{Ablation is mandatory}: Run Baseline, T-only, C-only, TC, Full configurations. Never trust single-configuration results.
    
    \item \textbf{Check component engagement}:
    \begin{itemize}
        \item Topology: Measure kNN preservation explicitly—should increase with $\lambda_T$
        \item Causality: Measure correlation between latent codes and nuisance factors—should decrease with $\lambda_C$
        \item Energy: Training loss should stabilize earlier; generalization gap should decrease
    \end{itemize}
    
    \item \textbf{If components don't engage} (e.g., 0\% topology preservation):
    \begin{itemize}
        \item First check implementation bugs
        \item Then consider domain mismatch (see Section~\ref{sec:tc_conflict})
        \item Don't force it—skip that component for your domain
    \end{itemize}
    
    \item \textbf{Marginal gains are OK}: Even 1-4pp improvements can be meaningful for high-stakes applications. But if gains are $<$1pp, question whether complexity is justified.
\end{enumerate}

\subsubsection{Common Pitfalls}

\textbf{1. Loss imbalance in classification tasks:}
\begin{itemize}
    \item \textbf{Symptom}: Training accuracy stuck at random chance; total loss is negative
    \item \textbf{Cause}: APS losses dominate classification loss
    \item \textbf{Fix}: Reduce $\lambda$ weights by 10--100×; add loss component clipping
\end{itemize}

\textbf{2. Assuming universal benefits:}
\begin{itemize}
    \item \textbf{Symptom}: Component shows 0\% improvement across all settings
    \item \textbf{Cause}: Domain mismatch (e.g., topology on low-dimensional data)
    \item \textbf{Fix}: Validate component engagement; skip if incompatible with your data
\end{itemize}

\textbf{3. Frozen representations with causality:}
\begin{itemize}
    \item \textbf{Symptom}: Causality component has no effect
    \item \textbf{Cause}: Gradient-based independence can't modify frozen embeddings
    \item \textbf{Fix}: Enable fine-tuning or use trainable encoders
\end{itemize}

\textbf{4. Insufficient batch size for topology:}
\begin{itemize}
    \item \textbf{Symptom}: Topology loss is noisy; preservation doesn't improve
    \item \textbf{Cause}: Too few samples for stable kNN graphs
    \item \textbf{Fix}: Increase batch size or compute kNN on full dataset offline
\end{itemize}

\subsubsection{Decision Tree}

\textbf{START: Do you have out-of-distribution (OOD) generalization concerns?}
\begin{itemize}
    \item \textbf{No} $\rightarrow$ Use standard architectures; APS likely unnecessary
    \item \textbf{Yes} $\rightarrow$ Continue ↓
\end{itemize}

\textbf{Q1: What's the strength of spurious correlations?}
\begin{itemize}
    \item \textbf{Weak ($<$80\%)} $\rightarrow$ Standard training likely sufficient
    \item \textbf{Medium (80--95\%)} $\rightarrow$ Use reconstructive architecture (implicit bias) + optional explicit regularization
    \item \textbf{Strong ($>$95\%)} $\rightarrow$ Explicit causal regularization (C component) critical
\end{itemize}

\textbf{Q2: Are your representations trainable?}
\begin{itemize}
    \item \textbf{Yes (learning from raw inputs)} $\rightarrow$ Full APS applicable
    \item \textbf{No (frozen pre-trained features)} $\rightarrow$ Skip causality (C); use topology (T) + energy (E) only
\end{itemize}

\textbf{Q3: Is your data high-dimensional with meaningful geometry?}
\begin{itemize}
    \item \textbf{Yes (images, audio, dense embeddings)} $\rightarrow$ Add topology (T)
    \item \textbf{No (low-dim, discrete, sparse)} $\rightarrow$ Skip topology (T)
\end{itemize}

\textbf{Q4: Do you have overfitting issues?}
\begin{itemize}
    \item \textbf{Yes} $\rightarrow$ Add energy (E) with small $\lambda_E$ (0.01--0.1)
    \item \textbf{No} $\rightarrow$ Energy optional but unlikely to hurt
\end{itemize}

\subsection{Summary of Boundary Conditions}

Our systematic investigation establishes that causal learning success depends on:

\begin{enumerate}
    \item \textbf{Architectural bias} (primary): Reconstruction $\gg$ explicit regularization in magnitude of effect
    \item \textbf{Spurious correlation strength}: Explicit methods critical only at $>$95\% correlation
    \item \textbf{Domain shift magnitude}: Weak shifts ($<$5\%) show minimal benefit from T+C; strong shifts ($>$10\%) show clear benefits
    \item \textbf{Representation learnability}: Frozen embeddings limit causality component effectiveness
    \item \textbf{Data modality}: High-dimensional continuous data benefits from topology; low-dimensional discrete data does not
\end{enumerate}

These boundary conditions reframe causal learning from a universal solution to a \textbf{targeted intervention}: apply the right regularization at the right time, and start with architectural choice.


APS can be seen as injecting domain-agnostic inductive biases
preservation, invariance to spurious factors, and data-driven energy
landscapes) that make learned representations more aligned with the true
data-generating factors. Our experiments on \textbf{MNIST} and \textbf{AG News}
demonstrated that APS yields latent maps where \textbf{neighborhoods are meaningful (T)},
\textbf{axes align with stable concepts (C)}, and \textbf{energy wells reinforce
rather than compete with geometric structure (E)}. These properties improve both
performance and explainability, with the key insight that
\textbf{constraints should be mutually reinforcing, deriving structure from
data rather than imposing arbitrary patterns}.

\textbf{Cross-Domain Validation:} Our NLP experiments on sentiment classification
across news topics revealed important \textbf{boundary conditions} for when domain
adaptation mechanisms provide benefits. While MNIST showed strong gains from
topology and causality components (70\% trustworthiness improvement, 77\% ARI
improvement), AG News with frozen pre-trained BERT embeddings exhibited minimal benefits
from T and C due to \emph{both} weak domain shift (2\% sentiment variance) \emph{and} frozen representations that prevent gradient-based regularization from restructuring features. However, TopologyEnergy provided
consistent regularization benefits against overfitting across \emph{both} settings, achieving a
negative generalization gap (-10.82pp) on AG News while maintaining OOD accuracy. The negligible OOD gain (+0.11pp, likely noise) indicates energy's primary value is overfitting prevention rather than OOD accuracy improvement. This contrast establishes that:
\begin{itemize}
    \item \textbf{Energy (E)} prevents overfitting consistently: effective capacity control regardless of shift strength or modality, but does not directly improve OOD accuracy
    \item \textbf{Topology (T) \& Causality (C)} benefits scale with shift magnitude and representation learnability
    \item Strong-shift scenarios with learnable representations (MNIST-style) showcase full APS potential
    \item Weak-shift scenarios with frozen features (AG News) benefit primarily from energy regularization
\end{itemize}

\textbf{Impact:} For the general ML community, APS offers a blueprint
for \textbf{geometric deep learning} in the latent space -- moving
beyond unstructured vector spaces to \emph{spaces with topology and
geometry tailored to the problem}. This resonates with the trend of
applying \textbf{differentiable constraints} (e.g. using TDA or
adversarial objectives) to ensure our models learn what we intend. APS
specifically could benefit LLMs by providing them with an internal
semantic atlas, potentially enabling better control (steering the model
towards certain regions yields certain types of generations) and more
predictable behavior. Similarly, in recommendation or personalization
systems, an APS embedding could help identify coherent user segments or
item categories through the energy basins, improving transparency and
fairness (as causal factors like demographic correlations could be
explicitly controlled in $Z$).

\textbf{Limitations and Future Work:} Our initial APS implementation
introduces several hyperparameters (the weights
$\lambda$'s, the choice of $k$, etc.) which require
tuning. In some cases, there may be trade-offs between the objectives --
e.g. perfect topology preservation might conflict with perfect
invariance if certain spurious features were part of local similarity in
data. Balancing these is non-trivial. Additionally, the current
formulation assumes we can either know or infer nuisance factors for the
causality loss; in truly unsupervised scenarios, one might use data
augmentations as a proxy (assuming certain transformations shouldn't
change $Z$). This could be further automated by techniques that learn
what to ignore (perhaps using attention mechanisms to attend to causal
features). 

The AG News results highlight a key data limitation: the dataset lacks ground-truth sentiment labels, requiring keyword-based pseudo-labeling. The resulting weak signal (54-72\% accuracy) and minimal domain shift (2\% sentiment variance) limit our ability to test T and C components. Attempted fine-tuning experiments confirmed that without proper sentiment annotations, models converge to random baseline (~50\% accuracy), validating the need for datasets with verified labels and stronger spurious correlations.

Future work should evaluate APS on:
\begin{enumerate}
    \item \textbf{Proper sentiment datasets:} Amazon Reviews (product category shift), Yelp (location/time shift) with verified sentiment labels
    \item \textbf{Strong domain shifts:} ColoredMNIST (90\%+ spurious correlation), Waterbirds, CivilComments with validated biases
    \item \textbf{Trainable representations:} Fine-tune or train from scratch on datasets with learnable spurious correlations
    \item \textbf{Multi-domain settings:} 5+ diverse domains where invariance learning is critical
    \item \textbf{Online learning:} Adapt APS for streaming data where domain shift evolves over time
\end{enumerate}

Another limitation is scalability: for extremely large
datasets, computing even approximate neighbor graphs is heavy -- one
might explore \emph{self-supervised contrastive approaches} to
approximate the topology loss (e.g. treat augmented pairs as neighbors).
Our TopologyEnergy formulation avoids the mode collapse issues typical of
traditional EBMs by deriving energy directly from data structure rather than
learning arbitrary patterns, making it more robust and scalable than
memory-based alternatives.

For future research, one exciting avenue is to extend APS to
\textbf{different geometries} (not just Euclidean latent spaces). For
instance, we could enforce topology and invariance while learning
embeddings on a \textbf{hyperbolic manifold} for inherently hierarchical
data -- combining APS with the Poincaré embedding
approach\cite{nickel2017poincare}.
Another direction is to incorporate \textbf{dynamic or temporal pattern
spaces} -- e.g. use APS for sequence models where the latent at each
time step forms an atlas of states (this might connect with state-space
models or neural ODEs that have
attractors\cite{dupont2019augmented}).
We also plan to investigate theoretical guarantees: under what
conditions does minimizing these losses recover the true generative
factors or the true manifold? Insights from recent identifiability
theory could guide this.

\subsection{Conclusion}

This work challenges the assumption that explicit causal regularization is universally necessary for out-of-distribution generalization. Through systematic experiments across vision, language, and synthetic domains, we establish clear \textbf{boundary conditions} for when different regularization strategies help.

Our central finding---that reconstructive architectures provide powerful implicit causal bias---reframes the causal learning problem: \textbf{architectural choice is primary}, with explicit regularizers serving as targeted, domain-specific corrections. On ColoredMNIST with 99\% spurious correlation, autoencoders achieve 82-86\% accuracy before any explicit regularization, with causal constraints adding only 0-4pp.

The Atlasing Pattern Space (APS) framework served as a diagnostic toolkit for dissecting these effects. Our experiments reveal that:
\begin{itemize}
    \item \textbf{Topology preservation} improves kNN fidelity in high-dimensional vision tasks but fails completely on low-dimensional synthetic data (0\% preservation)
    \item \textbf{Causal invariance} provides modest gains (+1-4pp) when spurious correlations are strong ($>$90\%) and representations are learnable
    \item \textbf{Energy regularization} consistently prevents overfitting but provides marginal OOD accuracy improvements
\end{itemize}

By reporting both successes and failures transparently, we provide practitioners with \textbf{decision criteria}: use topology for high-dimensional data with meaningful geometry; apply causality when facing strong spurious correlations with trainable representations; include energy for training stability. This honest characterization of boundary conditions is more scientifically valuable than claims of universal benefits.

The practical guidelines in Section~\ref{sec:practical_guidelines} synthesize these findings into actionable recommendations, including a decision tree, component selection matrix, and common pitfalls guide. These tools help practitioners avoid wasted effort applying regularization where it won't help.

Looking forward, key open questions remain: What are the theoretical limits of implicit causal bias? Can we predict a priori which domains will benefit from geometric regularization? How can we balance multi-objective losses in pure classification settings without the stability provided by reconstruction?

Ultimately, this work advocates for a more \textbf{nuanced, evidence-based approach} to causal learning: test assumptions, measure component engagement, and apply regularization as a targeted intervention rather than a universal solution. We hope these boundary conditions guide future research toward more realistic characterizations of when and why causal methods help.

% Include all references from the .bib file, not just those explicitly cited
\nocite{*}

\bibliographystyle{plain}
\bibliography{references}

\subsection*{Code Availability}

The implementation of the APS framework, including TopologyEnergy and all experimental code, is available at:
\begin{center}
\url{https://github.com/freemanhui/atlasing_pattern_space}
\end{center}

\end{document}
