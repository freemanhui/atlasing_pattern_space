\section{Discussion and Conclusion}\label{discussion-and-conclusion}

We presented \textbf{Atlasing Pattern Space (APS)}, a framework that
integrates principles from manifold learning, causal inference, and
energy-based modeling to produce \textbf{structured, interpretable
latent spaces} for LLMs and other high-dimensional models. A critical
contribution is the discovery that traditional memory-based energy functions
\emph{catastrophically fail} when combined with topology preservation,
leading to the development of \textbf{TopologyEnergy} -- a data-driven
energy formulation that achieves 902\% better label alignment than
memory-based approaches while maintaining reconstruction quality.

APS can be seen as injecting domain-agnostic inductive biases (local similarity
preservation, invariance to spurious factors, and data-driven energy
landscapes) that make learned representations more aligned with the true
data-generating factors. Our experiments on \textbf{MNIST} and \textbf{AG News}
demonstrated that APS yields latent maps where \textbf{neighborhoods are meaningful (T)},
\textbf{axes align with stable concepts (C)}, and \textbf{energy wells reinforce
rather than compete with geometric structure (E)}. These properties improve both
performance and explainability, with the key insight that
\textbf{constraints should be mutually reinforcing, deriving structure from
data rather than imposing arbitrary patterns}.

\textbf{Cross-Domain Validation:} Our NLP experiments on sentiment classification
across news topics revealed important \textbf{boundary conditions} for when domain
adaptation mechanisms provide benefits. While MNIST showed strong gains from
topology and causality components (70\% trustworthiness improvement, 77\% ARI
improvement), AG News with pre-trained BERT embeddings exhibited minimal benefits
from T and C due to weak domain shift (2\% sentiment variance) and inherent
topic-invariance in BERT representations. However, TopologyEnergy provided
consistent regularization benefits across \emph{both} settings, achieving a
negative generalization gap (-10.82pp) on AG News that validates its role as
a universal regularizer. This contrast establishes that:
\begin{itemize}
    \item \textbf{Energy (E)} benefits are universal: effective regularization regardless of shift strength or modality
    \item \textbf{Topology (T) \& Causality (C)} benefits scale with shift magnitude and representation learnability
    \item Strong-shift scenarios with learnable representations (MNIST-style) showcase full APS potential
    \item Weak-shift scenarios with frozen features (AG News) benefit primarily from energy regularization
\end{itemize}

\textbf{Impact:} For the general ML community, APS offers a blueprint
for \textbf{geometric deep learning} in the latent space -- moving
beyond unstructured vector spaces to \emph{spaces with topology and
geometry tailored to the problem}. This resonates with the trend of
applying \textbf{differentiable constraints} (e.g. using TDA or
adversarial objectives) to ensure our models learn what we intend. APS
specifically could benefit LLMs by providing them with an internal
semantic atlas, potentially enabling better control (steering the model
towards certain regions yields certain types of generations) and more
predictable behavior. Similarly, in recommendation or personalization
systems, an APS embedding could help identify coherent user segments or
item categories through the energy basins, improving transparency and
fairness (as causal factors like demographic correlations could be
explicitly controlled in $Z$).

\textbf{Limitations and Future Work:} Our initial APS implementation
introduces several hyperparameters (the weights
$\lambda$'s, the choice of $k$, etc.) which require
tuning. In some cases, there may be trade-offs between the objectives --
e.g. perfect topology preservation might conflict with perfect
invariance if certain spurious features were part of local similarity in
data. Balancing these is non-trivial. Additionally, the current
formulation assumes we can either know or infer nuisance factors for the
causality loss; in truly unsupervised scenarios, one might use data
augmentations as a proxy (assuming certain transformations shouldn't
change $Z$). This could be further automated by techniques that learn
what to ignore (perhaps using attention mechanisms to attend to causal
features). 

The AG News results highlight a key limitation: frozen pre-trained embeddings
limit the causality component's effectiveness since gradient-based independence
cannot modify input features. Future work should evaluate APS on:
\begin{enumerate}
    \item \textbf{Trainable embeddings:} Fine-tune BERT or train from scratch to allow full representation learning
    \item \textbf{Strong domain shifts:} ColoredMNIST (90\%+ spurious correlation), Waterbirds, CivilComments with validated biases
    \item \textbf{Multi-domain datasets:} 5+ diverse domains where invariance learning is critical
    \item \textbf{Online learning:} Adapt APS for streaming data where domain shift evolves over time
\end{enumerate}

Another limitation is scalability: for extremely large
datasets, computing even approximate neighbor graphs is heavy -- one
might explore \emph{self-supervised contrastive approaches} to
approximate the topology loss (e.g. treat augmented pairs as neighbors).
Our TopologyEnergy formulation avoids the mode collapse issues typical of
traditional EBMs by deriving energy directly from data structure rather than
learning arbitrary patterns, making it more robust and scalable than
memory-based alternatives.

For future research, one exciting avenue is to extend APS to
\textbf{different geometries} (not just Euclidean latent spaces). For
instance, we could enforce topology and invariance while learning
embeddings on a \textbf{hyperbolic manifold} for inherently hierarchical
data -- combining APS with the Poincaré embedding
approach\cite{nickel2017poincare}.
Another direction is to incorporate \textbf{dynamic or temporal pattern
spaces} -- e.g. use APS for sequence models where the latent at each
time step forms an atlas of states (this might connect with state-space
models or neural ODEs that have
attractors\cite{dupont2019augmented}).
We also plan to investigate theoretical guarantees: under what
conditions does minimizing these losses recover the true generative
factors or the true manifold? Insights from recent identifiability
theory could guide this.

In conclusion, Atlasing Pattern Space represents a step toward
\emph{geometry-aware, causally-informed representation learning}. By
unifying ideas across subfields (topological deep learning, causal ML,
energy-based memory
networks\cite{ramsauer2020hopfield}),
it provides a framework for building \textbf{latent spaces that are as
structured and rich as the data they model}. The discovery of TopologyEnergy's
superiority over memory-based approaches, combined with validation across
vision and language domains, establishes practical guidelines for when to
apply different APS components. We hope this sparks further
exploration into \textbf{structured embeddings} that can ultimately lead
to more generalizable and interpretable AI systems.
