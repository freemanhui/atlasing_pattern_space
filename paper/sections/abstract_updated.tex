\section{Abstract}\label{abstract}

Atlasing Pattern Space (APS) is a novel framework for learning
\textbf{structured latent representations} in large language models
(LLMs) and other high-dimensional domains. APS diverges from standard
embedding approaches by introducing explicit \textbf{geometric
constraints} during representation learning. In particular, APS enforces
three complementary properties in the latent space: \textbf{Topology
preservation (T)} -- similar patterns lie close together, preserving
neighborhood relations; \textbf{Causality awareness (C)} --
representations are invariant to nuisance factors and emphasize
underlying causal features using independence-promoting losses (e.g.
HSIC, IRM); and \textbf{Energy shaping (E)} -- through a novel
\textbf{TopologyEnergy function} that reinforces rather than competes
with topology preservation, addressing the catastrophic failure of
memory-based attractor approaches (which degraded label alignment by
92\%). The result is an \textbf{interpretable ``atlas'' of the pattern
space}: a latent manifold where local neighborhoods reflect true
similarity, axes align with stable, generalizable factors, and the
data-driven energy landscape provides regularization without compromising
semantic structure. We outline the APS framework, relate it to prior work
in manifold learning, causal representation learning, and energy-based
models, and present experimental validation on \textbf{MNIST} demonstrating that
TopologyEnergy achieves 902\% better label alignment (ARI) and 51.6\%
better trustworthiness compared to memory-based approaches while
maintaining reconstruction quality. \textbf{NLP validation} on AG News sentiment classification
across news topics reveals that TopologyEnergy provides effective regularization
(achieving a negative generalization gap of -10.82pp on out-of-distribution data)
even in weak domain shift scenarios where topology and causality components show
limited benefits due to pre-trained embedding invariance. The framework's applicability extends
to NLP, vision, and recommender systems, yielding representations that
are both \textbf{geometrically structured and highly interpretable},
enabling improved generalization and insightful visualization of learned
pattern spaces. Importantly, these results establish \textbf{boundary conditions}
for when complex domain adaptation mechanisms are warranted, demonstrating that
energy-based regularization provides universal benefits while topology and
causality excel in strong-shift scenarios with learnable representations.
